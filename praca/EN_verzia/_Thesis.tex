\documentclass[a4paper,11pt,openany]{book} % bez prazdnych stran
%\documentclass[a4paper,11pt]{book} % s prazdnymi stranami, aby kapitola zacinala na neparnej, vhodne pre tlac

% definovanie kompilatora (magic comment):
% !TeX program = xelatex

% ============== PREAMBULA ==============
% (nacitanie balikov a podobne)
% ============== PREAMBULA ==============
% (nacitanie balikov a podobne)
% Ctrl+klik na balik zobrazi dokumentaciu k baliku (v TeXstudiu)
% Je tu vela uzitocnych balikov ak ale nejaky nepotrebujete, zakomentujte ho


% geometria a formatovanie
\usepackage[left=2cm,right=2cm,top=0cm,bottom=0cm, paperwidth=700mm, paperheight=1000mm]{geometry}
%\usepackage{a0poster}
\usepackage{multicol}
 
\columnseprule=2pt % Hrubka oddelovaca medzi stlpcami
\columnsep=100pt % Medzera medzi oddelovacmi stlpcov a textom


% jazyk
\usepackage[main=english,slovak]{babel}  % primarny jazyk: EN, dalsie jazyky: SK


% bibliografia
\usepackage[style=iso-numeric,backend=bibtex,giveninits=true,sorting=nyt]{biblatex}
\addbibresource{literatura.bib}


% fromatovanie
\usepackage{indentfirst} % odsadenie prveho odstavca
\usepackage{enumitem} % číslovanie ((a),(b),(c),... i), ii), iii),...)
\usepackage[small]{caption} % popisy obrazkov


% grafika
\usepackage{graphicx} % obrazky
%\usepackage{tikz}
\usepackage{subcaption} % podobrazky (subfigure)
\graphicspath{ {./figures/} } % priecinok s obrazkami
\usepackage{wrapfig} % obrazky obtekane textom
\usepackage{xcolor} % farby
\usepackage{colortbl} % farebne tabulky
\usepackage{multirow} % treba pre merge-ovanie budniek v tabulkach

% matematika
\usepackage{mathtools} % rozsirenie zakladneho matematickeho balika amsmath
\usepackage{amssymb} % dalsie symboly a fonty (napriklad pre mnozinu realnych cisel)
\usepackage[locale=DE]{siunitx} % jednotky SI

% Definície, Vety, Dôkazy...
\usepackage{amsthm}
\theoremstyle{definition}
\newtheorem{thm}{Veta}[section]
\newtheorem{defn}[thm]{Definícia}
\newtheorem{lem}[thm]{Lemma}
\newtheorem{cor}[thm]{Dôsledok}
\newtheorem{rem}[thm]{Poznámka}
\newtheorem{exmp}[thm]{Príklad}


% hyperreferencia
\usepackage[hidelinks]{hyperref} % hyperreferencia


% ============== DOKUMENT ==============
\begin{document}
	% ====== Uvodne casti ======
	\pagestyle{empty}
	% Do tlacenej verzie v kniznej vazbe sa obalka neviaze, sluzi ako predloha pevnej obalky. Do tepelnej vazby sa viaze.

\begin{center}
	\textsc{\LARGE
		Slovak University of Technology in Bratislava\\
		\vspace{10pt}
		Faculty of Civil Engineering}\\
	
	\begin{flushleft}
		Reg. No.: SvF-
	\end{flushleft}
	\vfill
	
	\textsc{\LARGE
		Thesis Title}\\ 
	\vspace{10pt}
	{\Large
		Bachelor's/Master's/PhD thesis}
\end{center}

\vfill

{\Large
	\noindent \textbf{2022}
	\hfill \textbf{Author's Name} % pripadne tituly
}

\ifbool{printVersion}{ % prazdna strana
	\newpage
	\
	\newpage
}
\
	\setcounter{page}{1}
	\begin{center}
	\textsc{\LARGE
		Slovak University of Technology in Bratislava\\
		\vspace{10pt}
		Faculty of Civil Engineering}\\
	
	\vfill
	
	\textsc{\LARGE
		Thesis Title}\\ 
		\vspace{10pt}
	{\Large
		Bachelor's/Master's/PhD thesis}
\end{center}

\vfill

\noindent
\begin{tabular}{ll}
    Study programme: & Mathematical and Computer Modelling\\
	Study field: & 9.1.9. Applied Mathematics\\
	Training workplace:& Department of Mathematics and Constructive Geometry\\
	Supervisor: & Ing. Mgr. Lukáš Tomek, PhD. \\
	% Consultant: & Name, Surname and Academic titles \\
\end{tabular}

\vspace{0.1\textheight}

{\Large
	\noindent \textbf{Bratislava 2022}
	\hfill \textbf{Author's Name} % pripadne tituly
}

\ifbool{printVersion}{ % prazdna strana
	\newpage
	\
	\newpage
}
\


	\includepdf[pages=-,pagecommand={}]{documents/zadanie_AIS.pdf} % stiahnete z AIS
	% Pre vlozenie pokynov na vypracovanie, odkomentujte nasledujuce 3 riadky
%	\ifbool{printVersion}{\cleardoublepage}
%	
%	\includepdf[pages=-,pagecommand={}]{documents/Guidelines_Bc.pdf}
	\null
\vfill
\thispagestyle{empty}

\subsection*{Declaration}

I declare that this thesis has been composed solely by myself under supervision of my supervisor and using the literature stated in Bibliography.
\vspace{10pt}

\noindent Bratislava 6. 5. 2022 \hfil
\newline

\begin{flushright}
	Author's Name $\qquad$
\end{flushright}
\newpage

\ifbool{printVersion}{ % prazdna strana
	\thispagestyle{empty}
	\
	\newpage
}
\
	\include{Acknowledgement}
	\pagestyle{plain}
	\thispagestyle{empty}

\section*{Abstract}
\noindent \textbf{Title:} Názov práce v angličtine\\
\textbf{Abstract:} Abstrakt má byť krátky, presný a zrozumiteľný text, ktorým predstavujete svoju prácu. Píše sa ako jeden odsek a mal by mať zhruba 50 -- 300 slov. Najlepšie je písať ho nakoniec, keď už má človek všetko dobre uležané v hlave a venovať mu dostatočnú pozornosť. Pri odovzdaní práce sa slovenský aj anglický Abstrakt zadáva aj do AIS a dá sa chápať ako \uv{upútavka} na vašu prácu.

\vspace{10pt}

\noindent \textbf{Keywords:} 3 až 5 kľúčových slov/slovných spojení oddelených čiarkou

\vspace{+20pt}



\section*{Abstrakt}

\noindent \textbf{Názov práce:} Názov práce v slovenčine\\
\textbf{Abstrakt:} Preklad abstraktu do Slovenčiny.

\vspace{10pt}

\noindent \textbf{Kľúčové slová:} 3 až 5 kľúčových slov/slovných spojení oddelených čiarkou
%\newpage

\cleardoublepage


	\include{Preface}
	\tableofcontents
	
%	\include{Zoznam_priloh}
%	\include{Zoznam_skratiek}

	% ====== Jadro prace ======
	\chapter{Introduction}

Úlohou Introduction je uviesť čitateľa do problematiky práce. Ak ste nepísali Preface, tak na tomto mieste môžete popísať ciele práce. Obvykle sa tu nezachádza do detailov teórie a nebýva tu veľa vzorcov. Môžete tu prípadne vysvetliť základné pojmy, s ktorými budete narábať.

Je tu priestor na charakterizáciu stavu poznania v oblasti, ktorá je predmetom záverečnej práce, citovanie literatúry (knihy, vedecké články, iné záverečné práce) ktorá rieši podobnú problematiku, uviesť, z akých zdrojov vychádzate alebo na ne priamo nadväzujete. (Toto je veľmi dôležité v dizertačnej práci.)

Ak ste nepísali Preface, tak je dobré  na konci Introduction stručne povedať o členení práce -- o čom sú jednotlivé kapitoly, prípadne sekcie práce.

	\chapter{V akom editore písať? Aký kompilátor používať?}\label{sec:aky_editor}

\subsubsection{Editor}
Existuje niekoľko populárnych LaTeX editorov: TeXstudio, TeXmaker, TeXnicCenter, LyX, Overleaf (ten je online),... Dobrý prehľad nájdete \href{https://beebom.com/best-latex-editors/}{napríklad tu}. \textbf{Odporúčam TeXstudio}, dá sa v ňom pracovať veľmi efektívne. V akomkoľvek editore sa naučte \textbf{používať klávesové skratky} (v TeXstudiu ich nájdete v Options/Configure TeXstudio/Shortcuts), rýchlo sa orientovať a preklikávať v TeX súbore pomocou referencií, preklikávať sa medzi TeX súborom a pdf, neuveriteľne to zrýchli prácu.

Ak sa vám páči mať vo veciach systém a poriadok, tak dobrý nápad je priebežne prácu commitovať do repozitára na \href{https://github.com/}{GitHub}-e. Mimochodom, GitHub je veľmi užitočný pri programovaní -- hlavne ak s niekým spolupracujete (napríklad so školiteľom) alebo píšete komplexnejší program (v rámci záverečnej práce to tak často býva). Ak GitHub nepoužívate, odporúčam začať.

\subsubsection{Kompilátor}
Akýkoľvek editor si zvolíte, tak na kompilovanie odporúčam \textbf{kompilátor XeLaTeX}. Výsledné PDF bude mať správne kódovanie slovenských znakov\footnote{Vďaka tomu, že XeLaTeX používa Unicode.} (v texte sa bude dať dobre vyhľadávať). Pre väčšinu editorov (aj TeXstudio) sa dá kompilátor nastaviť aj priamo v tex súbore pomocou tzv \uv{magic commentu}, v súbore \verb|_Zaverecna_praca.tex| je to riadok \verb|% !TeX program = xelatex|, ktorý editoru hovorí, že tento dokument má kompilovať pomocou kompilátora XeLaTeX. Samozrejme, v TeXstudiu je možné aj prenastaviť defaultný kompilátor pre všetky dokumenty: Options/Configure TeXstudio/Build/Default compiler/XeLaTeX.

Ak ale chcete kompilovať pomocou \textbf{kompilátora PdfLaTeX}, treba správne kódovanie v PDF zabezpečiť pomocou pridania balíkov do preambuly. Funguje napríklad táto kombinácia (ale sú aj iné možnosti):
\begin{verbatim}
	\usepackage[utf8]{inputenc} % input encoding
	\usepackage[T1]{fontenc} % spravne makcene
	\usepackage{lmodern} % Latin modern fonts
\end{verbatim}
V súbore \verb|_Zaverecna_praca.tex| treba vymazať magic comment \verb|% !TeX program = xelatex| (ak defaultne používate PdfLaTeX) resp. ho prepísať na \verb|% !TeX program = pdflatex|.

\subsubsection{Porovnávanie verzií (diff)}
Keď školiteľovi posielate novú verziu, je veľmi užitočné mu v nej vyznačiť zmeny oproti poslednej verzii, ktorú čítal. Odporúčam na to výborný skript \textbf{latexdiff} napísaný v jazyku Perl. Je to analógia funkcie Track changes vo Worde, ktorú možno poznáte. Skript sa inštaluje napríklad takto:
\begin{enumerate}
	\item Nainštalujte latexdiff. Napríklad cez MiKTeX Console (Run as administrator), Packages.
	\item Nainštalujte \href{https://strawberryperl.com/}{Strawberry Perl}.
\end{enumerate}
Keď už máte skript nainštalovaný, používa sa veľmi jednoducho a rýchlo (keď už sa to naučíte):
\begin{enumerate}
	\item V príkazovom riadku sa nastavíte do správneho priečinka (rýchly spôsob v TeXstuidu je Tools/Open External Terminal, a tam prípadne príkaz \verb|cd ..| na posun o priečinok vyššie).
	\item V príkazovom riadku treba spustiť príkaz typu
	\begin{verbatim}
		latexdiff --flatten "stara_verzia.tex" "nova_verzia.tex" > "diff.tex"
	\end{verbatim}
	Vstupom je stará verzia dokumentu \verb|stara_verzia.tex|, nová verzia \verb|nova_verzia.tex| a výstupom je súbor \verb|diff.tex|, v ktorom sú vyznačené zmeny. Nastavenie \verb|--flatten| je potrebné, keď sa dokument skladá z viacerých súborov (ako táto šablóna). Použitie skriptu na túto šablónu môže vyzerať takto
	\begin{verbatim}
		latexdiff --flatten "SK_verzia_stara\_Zaverecna_praca.tex" "SK_verzia\
		_Zaverecna_praca.tex" > "Porovnanie_verzii\diff.tex"
	\end{verbatim}
	V príkazovom riadku teda musíte byť v priečinku \verb|ZaverecnaPracaMPM\praca|, v ktorom sa nachádza priečinok so starou verziou\footnote{Na uchovávanie starých verzií aj tu výborne poslúži GitHub, ale ak ho nepoužívate, tak si priebežne zálohujte celý priečinok s prácou.} \verb|SK_verzia_stara| aj s novou verziou \verb|SK_verzia| a musí byť vytvorený priečinok \verb|Porovnanie_verzii|. Ak máte iné umiestnenia, cesty si upravte. Pokiaľ nemáte v názvoch priečinkov a súborov medzery, tak úvodzovky v príkazoch nie sú potrebné.
	\item Do priečinka \verb|Porovnanie_verzii| dajte všetky obrázky a dokumenty (podpriečinky \verb|figures| a \verb|documents|), ktoré sa do práce vkladajú a aj bibliografický \verb|.bib| súbor (všetko z novej verzie).
	\item Súbor \verb|diff.tex| skompilujete, čím vytvoríte výsledné \verb|diff.pdf|, ktoré môžete spolu s aktuálnou verziou poslať školiteľovi.
\end{enumerate}








	\chapter{Členenie práce}

Keď začínate prácu písať, je dobré urobiť si základnú kostru -- štruktúru kapitol, podkapitol,... Premyslieť si a napísať krátku poznámku, o čom sa v časti bude hovoriť a až potom začať jednotlivé časti postupne napĺňať. Samozrejme, členenie sa časom bude upravovať, ale je dôležité mať na začiatku nejakú predstavu.

Okrem základného členenia (\texttt{\textbackslash chapter}, \texttt{\textbackslash section}, \texttt{\textbackslash subsection},...) sa v špecifických prípadoch môže hodiť rozdelenie na časti pomocou \texttt{\textbackslash part} (o úroveň nad \texttt{\textbackslash chapter}).


\section{Section}\label{}
Blabla

\subsection{Subsection}
Blabla

\subsubsection{Subsubsection}
Blabla

%\paragraph{Paragraf}
%Blabla
%
%\subparagraph{Subparagraf}
%Blabla
	\include{Chapter3}
	\chapter{Conclusions}

V Conclusions zhrniete, čomu sa venovala vaša práca, ako sa vám podarilo naplniť stanovené ciele a k akým výsledkom ste prišli. Môžete zhodnotiť váš postup riešenia problému, jeho výhody/nevýhody. V Conclusions je vhodné aj naznačiť, akými smermi by sa dalo v práci pokračovať, aké zostali nevyriešené otázky, kde vidíte možnosti vylepšenia a podobne.

	\thispagestyle{empty}

\chapter*{Resumé}
\addcontentsline{toc}{chapter}{Resumé}

Pre práce v cudzom jazyku treba napísať resumé (zhrnutie) v slovenčine, v ktorom sa zhrnie, o čom je práca, aké boli jej ciele, použité metódy, základné myšlienky, čo sa podarilo dosiahnuť a podobne.

\cleardoublepage



	% ====== Bibliografia a prilohy ======
%	\nocite{*} % Vypise v bibliografii aj zroje, ktore neboli citovane
	\printbibliography[heading=bibintoc]
	
	\include{Use_of_AI}
	
	%\includepdf[pages=-,pagecommand={}]{documents/priloha1.pdf} % pripadne prilohy
\end{document}